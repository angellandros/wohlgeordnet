\documentclass[11pt,a4paper,]{article}
\usepackage[utf8]{inputenc}

\usepackage{amsfonts}
\usepackage{amssymb}
\usepackage{amsthm}
\usepackage{amsmath}

\usepackage{color}
\usepackage{hyperref}

\usepackage{xepersian}
\usepackage{fontspec}

\usepackage{fullpage}

\settextfont[Scale=0.95,
 BoldFont={persian-modern-bold.otf}, 
 BoldItalicFont={persian-modern-boldoblique.otf},
 ItalicFont={persian-modern-oblique.otf}
 ]{persian-modern-regular.otf}
\setdigitfont{persian-modern-regular.otf}

\title{\textbf{برهان بر این که هر مجموعه را می‌توان خوش‌ترتیب کرد \\
(از نامه‌یي به آقای هیلبرت)}}
\author{ارنست تسرملو در گوتینگن \\ برگردان: محمدعلی اعرابی}
\date{خرداد ۱۳۹۷}

\begin{document}

\maketitle

برهاني که در ادامه خواهد آمد حاصل صحبتي است که من هفته‌ی گذشته با آقای ارهارد اشمیت داشتم.

\begin{enumerate}
\item 
بگذاریم 
$M$
مجموعه‌یي با توانمندی%
\footnote{\lr{Mächtigkeit}}
[= کاردینالیتی]
$\mathfrak{m}$
باشد،
که عضوهایش را با
$m$
می‌نمایانیم،
$M'$
با توانمندی
$\mathfrak{m}'$
یک زیرمجموعه‌اش، 
که باید دست کم یک عضو
$m$
بدارد،
ولی نیز می‌تواند همه‌ی عضوهای
$M$
را بگنجاند،
و بگذارید
$M - M'$
مجموعه‌ی مکمل%
\footnote{\lr{komplementäre Teilmenge}}
نسبت به 
$M'$
باشد.
دو زیرمجموعه متفاوت پنداشته می‌شوند
اگر هر یک عضوي بدارند که در دیگری نیست.
مجموعه‌ی همه‌ی زیرمجموعه‌های مانند
$M'$
با
$\mathsf{M}$
نموده می‌شود.

\item
\textit{هر زیرمجموعه مانند 
$M'$
را می‌توان با یک عضو مانند
$m_1'$
متناظر دانست،
که در خود
$M'$
گنجانده است و می‌توان آن را عضو
برجسته‌ی
$M'$
نامید.}
این‌چنین یک تناظر
$\gamma$
میان مجموعه‌ی 
$\mathsf{M}$
و عضوهای مجموعه‌ی 
$M$
به دست می‌آید.
شمار چنین تناظرهایي برابر است با حاصل ضرب
$\prod \mathfrak{m}'$
روی همه‌ی زیرمجموعه‌های مانند
$M'$
و به هر رو با
$0$
نابرابر است.
در ادامه یک تناظر
$\gamma$
از بنیان ساخته شده‌است و به این روی یک 
خوش‌ترتیبی روی عضوهای 
$M$
به دست آمده‌است.

\item
\textit{تعریف.}


\end{enumerate}

\end{document}
